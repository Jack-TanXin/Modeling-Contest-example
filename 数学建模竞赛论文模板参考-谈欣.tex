% ------------------------------------------------------ %
% 由于很多参加数学建模竞赛的同学没有\LaTeX基础,因此做
% 本模板用于各位写手快速上手写作,不要追究原理,先把模
% 板用会再说。PS:上交附件时请把这部分注释全部删掉,作
% 品出现任何信息都会判为作弊,这里出现我的信息了。
% ------------------------------------------------------ %
%         谈欣 如有错误请联系1792733991@qq.com
%         谈欣 如有错误请联系1792733991@qq.com
%         谈欣 如有错误请联系1792733991@qq.com
% ------------------------------------------------------ %
%     更多内容关注:
%     个人网页:https://jack-tanxin.github.io
%     个人Github主页:https://github.com/Jack-TanXin
% ------------------------------------------------------ %
% 数学建模论文是个结构非常复杂的论文,因此使用\LaTeX必须
% 编译数次——无论你使用哪种发行版或编辑器,原因详见\LaTeX
% 的编译原理,一般写好以后编译2次就是文章真正的面目。
% ------------------------------------------------------ %
%              请使用XeLaTeX进行编译!!!
%              请使用XeLaTeX进行编译!!!
%              请使用XeLaTeX进行编译!!!
% ------------------------------------------------------ %
\documentclass[UTF8,12pt]{ctexart}

% ------------------------------------------------------ %
% 下面的包基本上都是数学建模必需的包,没有这些包我想基
% 本上是写不出完整的论文的,仅供大家参考
% ------------------------------------------------------ %
\usepackage[a4paper,left=2.5cm,right=2.5cm,top=2.5cm,bottom=2.5cm]{geometry}    % 这个包专门用来调整页边距
\usepackage{amsmath}    % 数学包
\usepackage{fancyhdr}   % 页眉页脚设定

% 插入图片的包
\usepackage{graphicx}
\usepackage{subfigure}

% 插入表格的包
\usepackage{float}
\usepackage{booktabs}

% 删去图注、表注的冒号
\usepackage{caption}
\DeclareCaptionLabelSeparator{twospace}{\ ~}   %%这三条语句即可
\captionsetup{labelsep=twospace}

% 颜色包,定义颜色,会在附录中使用到
\usepackage{xcolor}
\definecolor{gray}{RGB}{128,128,128}

% 链接包:这个宏包在论文中没有任何作用,单纯为了方便给我做广告
% 让目录可以索引到正文,所以这个包不能删!
\usepackage{hyperref}
\hypersetup{colorlinks=true,linkcolor=black,citecolor=blue,filecolor=black,urlcolor=cyan,pdftitle={Overleaf Example},pdfpagemode=FullScreen}
    
% 附录插入代码
\usepackage{listings}
\lstset{basicstyle=\ttfamily,columns=fullflexible,frame=single,breaklines=true,keywordstyle=\color{blue},commentstyle=\color{gray}}

% 务必注意:图片在figures目录下
% 务必注意:图片在figures目录下
% 务必注意:图片在figures目录下
\graphicspath{{figures/}} 



\begin{document}   % 文章开始



% ----------------------------------------------- %
%              请在这里输入文章标题
% ----------------------------------------------- %
%              请在这里输入文章标题
% ----------------------------------------------- %
\title{数学建模内容参考模板}



\author{} % 不要动!!
\date{} % 不要动!!
\maketitle % 不要动!!
\vspace{-4em}  % 不要动!!



% ---------------------------------------------------- %
%  打开chapter文件夹里的摘要.tex文件,编辑其中的内容
% ---------------------------------------------------- %
%  打开chapter文件夹里的摘要.tex文件,编辑其中的内容
% ---------------------------------------------------- %
\begin{abstract}

    摘要是全文最重要的部分,也是阅卷人首先看到的部分,阅卷人会只根据摘要将文章分成三六九等,所以如果不认真写摘要的话你就会有大麻烦,请务必留至少两小时用于摘要的打磨,将其控制在1面!!!
    
    针对问题一:在写摘要的时候请搞清楚,首先摘要的第一段是题设的背景,你需要随便胡扯几句,但别扯太多,毕竟要把摘要控制在一面很难!然后写完第一段后从第二段开始就要开始介绍你对每个题目的理解、过程以及求解与评价,语言尽量精炼,不要啰嗦,再强调一次:那么多内容的摘要压缩在一面非常难!下面我将给大家演示如何对于自己已经完成的“模型建立与求解”部分进行摘要描述。
    
    针对问题二,本文基于可视化和假设检验对所给数据集进行数据分析。首先对产品的销售价格和销售量的关系进行探究,本文分别通过\textbf{斯皮尔曼相关系数}对相关性进行定量描述,求得$\rho$=-0.2946,反映出\textbf{销售价格和销售量的相关性较弱}。接着对区域与销量的关系进行探究,通过\textbf{方差检验}得知\textbf{不同地区对订单的需求量有显著差异},并通过直方图探究出不同区域产品的需求量的不同特性。然后,本文对产品的销售方式与需求量的关系进行探究,通过\textbf{Mann-Whitney U}检验得出\textbf{线上销售与线下销售的销售量存在显著差异}。最后,通过\textbf{单样本Wilcoxon符号秩检验}将时间序列整体与促销活动单日的销量相比较,得出\textbf{促销活动单日的销量与平时的销量有显著差异}。为了将各特征的分布以及定量分析所得出的结论直观化,本文利用小提琴图、箱线图、直方图等对相关特征进行可视化,结果与定量分析一致。
    
    针对问题三,摘要内容摘要内容摘要内容摘要内容摘要内容摘要内容摘要内容摘要内容摘要内容摘要内容摘要内容摘要内容摘要内容摘要内容摘要内容摘要内容摘要内容摘要内容摘要内容摘要内容摘要内容摘要内容摘要内容摘要内容摘要内容摘要内容摘要内容摘要内容摘要内容摘要内容摘要内容摘要内容摘要内容摘要内容摘要内容摘要内容摘要内容摘要内容摘要内容摘要内容摘要内容摘要内容摘要内容摘要内容摘要内容摘要内容摘要内容摘要内容摘要内容摘要内容摘要内容摘要内容摘要内容摘要内容摘要内容摘要内容摘要内容。
    
    第二部分分别是针对数据分析的任务和机器学习的任务,可以看到摘要第一句话用于概括整个小题大致是怎么做的,从第二句话开始展开每一步,用“首先”、“接着”、“然后”、“最后”依次描述。在摘要中务必要把你的结果和评价情况直接呈现出来并且加粗,同时你使用的关键方法也需要加粗,通过这个源文件相信你已经知道该如何加粗了。
    
    \vspace{1em} % 移除2个行距的空白:这句代码千万不能动!!!移除2个行距的空白:这句代码千万不能动!!!
    
    \noindent{\textbf{关键词:}随机森林;方差选择法;Voting Classifier;层次聚类分析;决策树}
    
\end{abstract} 



\thispagestyle{empty}  % 清除摘要页的页码,不要动!!
\newpage  % 另开一页写目录,不要动!!



% ------------------------------------------------------------ %
%    这一部分自动生成目录,同时清除目录页的页眉和页脚,
%    清除目录页的页码以保证正文的第一面为页码1
%    并新开一面,如果不想要目录直接删除这部分
% ------------------------------------------------------------ %
\pagestyle{fancy}
\fancyhead{} % 页眉清空
\fancyfoot{} % 页脚清空
\begin{center} \tableofcontents \end{center} 
\thispagestyle{empty}\newpage



% ------------------------------------------------------------ %
%    设置正文的页眉、页脚,页眉没有内容,页码在页脚中间
%    并将本页设置为第一页
%    我浅浅用页眉给自己做了个广告,大家记得删除啊!
% ------------------------------------------------------------ %
\pagestyle{fancy} % 不要动!!
% 这是个广告,正式使用时请删除这一行
\fancyhead[R]{\href{https://jack-tanxin.github.io}{进入我的个人网页}} 
% 删除上一行后将下面的这句\fancyhead[R]{}前面的百分号删除就行
% \fancyhead[R]{}
\fancyhead[C]{} % 页眉中间清空 不要动!!
\fancyhead[L]{} % 页眉左侧清空 不要动!!
\fancyfoot[R]{} % 页脚右侧清空 不要动!!
\fancyfoot[C]{- {\thepage}\ -} % 页脚中间为页码 不要动!!
\fancyfoot[L]{} % 页脚左侧清空 不要动!!
\setcounter{page}{1}  % 不要动!!



% ---------------------------------------------------------- %
% 接下来的部分即为文章每个章节的内容,我这里使用的是2022年
% 国赛的论文,当时是有四道题目,每次竞赛的题目量都不一定,
% 请酌情更改,另外有时候可能会多一个灵敏度分析或是其他创新
% 的内容,记住一个准则:有多少个章节就在chapter文件夹中制作
% 多少个tex文件,每个文件夹管一个章节就行啦!!!
% ---------------------------------------------------------- %



% ---------------------------------------------------------- %
%    打开chapter文件夹里的问题重述.tex文件,编辑其中的内容
% ---------------------------------------------------------- %
%    打开chapter文件夹里的问题重述.tex文件,编辑其中的内容
% ---------------------------------------------------------- %
\section{问题重述}

\subsection{问题背景}

数学建模比赛论文是要我们解决一道给定的问题,所以正文部分一般应从问题重述开始,一般确定选题后就可以开始写这一部分了。
这部分的内容是将原问题进行整理,将问题背景和题目分开陈述即可,所以基本没啥难度。
本部分的目的是要吸引读者读下去,所以文字不可冗长,内容选择不要过于分散、琐碎,措辞要精练。
注意:在写这部分的内容时,绝对不可照抄原题!(论文会查重)
应为:在仔细理解了问题的基础上\cite{ref1},用自己的语言重新将问题描述一遍。语言需要简明扼要,没有必要像原题一样面面俱到。


\subsection{问题提出}

看到上面插入的“\cite{ref1}”了吗,这是在引用后面的文献,所以大家在文中引用文献的时候也像我这样在句子结束后面加一个这个就行了,他是蓝色的,比较醒目,一般也都是蓝色的好像。

现有一组由专家给出的关于玻璃的数据,要求通过分析与建模解决下面若干问题:

\textbf{问题1:} 问题提出内容问题提出内容问题提出内容问题提出内容问题提出内容问题提出内容问题提出内容问题提出内容。

\textbf{问题2:} 问题提出内容问题提出内容问题提出内容问题提出内容问题提出内容问题提出内容问题提出内容问题提出内容问题提出内容。

\textbf{问题3:} 问题提出内容问题提出内容问题提出内容问题提出内容问题提出内容问题提出内容问题提出内容问题提出内容问题提出内容。

\textbf{问题4:} 问题提出内容问题提出内容问题提出内容问题提出内容问题提出内容问题提出内容问题提出内容问题提出内容问题提出内容。



% ---------------------------------------------------------- %
%    打开chapter文件夹里的问题分析.tex文件,编辑其中的内容
% ---------------------------------------------------------- %
%    打开chapter文件夹里的问题分析.tex文件,编辑其中的内容
% ---------------------------------------------------------- %
\section{问题分析}

\subsection{问题一的分析}

从实际问题到模型建立是一种从具体到抽象的思维过程,问题分析这一部分就是沟通这一过程的桥梁,因为它反映了建模者对于问题的认识程度如何,也体现了解决问题的雏形,起着承上启下的作用,也很能反应出建模者的综合水平。

这部分的内容应包括:题目中包含的信息和条件,利用信息和条件对题目做整体分析,确定用什么方法建立模型,一般是每个问题单独分析一小节,分析过程要简明扼要, 不需要放结论。

建议在文字说明的同时用图形或图表(例如流程图)列出思维过程,这会使你的思维显得很清晰,让人觉得一目了然。


\subsection{问题二的分析}

从这个地方开始涉及到图片的插入了,下面是图片插入的方法,直接复制粘贴,修改几个参数就行了,没有 \LaTeX 基础的同学不要自己擅自修改,不然的话就可能有问题!下面这个流程图是我队友凯哥国赛时画的,当时他画的不太好看,现在已经画的很好看了哈哈哈哈。

\begin{figure}[H] % 这个H不要动!
	\centering % 不要动!
    % 在主文件中已经规定了图片全部放在figures文件夹里,这里意思是插入figures文件夹中的1.png图片文件,width是图片比例,0.95蛮好!
	\includegraphics[width=0.95\textwidth]{1.png} 
    % 最终文档中希望显示的图注
	\caption{流程图} 
    % 用于文内引用的标签,这个东西没什么用,但最好就是你按照文中的顺序把图片依次命名为1、2...,然后这个标签就跟着命名走
	\label{Fig.main1} 
\end{figure}

\subsection{问题三的分析}

问题分析内容问题分析内容问题分析内容问题分析内容问题分析内容问题分析内容问题分析内容问题分析内容问题分析内容问题分析内容问题分析内容问题分析内容问题分析内容问题分析内容问题分析内容问题分析内容问题分析内容问题分析内容问题分析内容问题分析内容问题分析内容问题分析内容问题分析内容问题分析内容问题分析内容问题分析内容问题分析内容问题分析内容问题分析内容问题分析内容问题分析内容


\subsection{问题四的分析}

问题分析内容问题分析内容问题分析内容问题分析内容问题分析内容问题分析内容问题分析内容问题分析内容问题分析内容问题分析内容问题分析内容问题分析内容问题分析内容问题分析内容问题分析内容问题分析内容问题分析内容问题分析内容问题分析内容问题分析内容问题分析内容问题分析内容问题分析内容问题分析内容问题分析内容问题分析内容问题分析内容问题分析内容问题分析内容问题分析内容问题分析内容





% ---------------------------------------------------------- %
%    打开chapter文件夹里的模型假设.tex文件,编辑其中的内容
% ---------------------------------------------------------- %
%    打开chapter文件夹里的模型假设.tex文件,编辑其中的内容
% ---------------------------------------------------------- %
\input{chapter/4-模型假设.tex}



% ---------------------------------------------------------- %
%    打开chapter文件夹里的符号说明.tex文件,编辑其中的内容
% ---------------------------------------------------------- %
%    打开chapter文件夹里的符号说明.tex文件,编辑其中的内容
% ---------------------------------------------------------- %
\input{chapter/5-符号说明.tex}



% ------------------------------------------------------------ %
%    打开chapter文件夹里的问题一建模.tex文件,编辑其中的内容
% ------------------------------------------------------------ %
%    打开chapter文件夹里的问题一建模.tex文件,编辑其中的内容
% ------------------------------------------------------------ %
\section{问题一的模型建立与求解}

%\subsection{数据预处理}

\subsection{模型建立}

模型建立部分是需要有大量的数学公式的,而很多小伙伴都没有 \LaTeX 基础,这里就不得不提到Mathtype了,队友可以先用Mathtype在Word或者WPS中敲好公式,然后再用Mathtype直接将公式转换为Tex代码。Mathtype安装、配置方法可以在我的个人网页中找到哈,仅供学习使用,不可用作商业用途!

值得注意的是这样转换而来的代码还是需要修改的,但是这个修改很容易!!!首先要注意,转换出来的代码全都是用:

\begin{lstlisting}
\[数学公式Tex代码\]
\end{lstlisting}

当然不排除我的Mathtype版本不对,正版、新版的已经修正。如果这样的话你需要把行内公式改为用一前一后的一对“\$”包裹,而行间公式则是用专门的数学环境——equation环境包裹。行内公式和行间公式示例如下:

\begin{lstlisting}
% Mathtype转化出的行内数学公式Tex代码
本部分使用\[\dfrac{4}{5}\]的数据集作为训练集……

% 正确修改为
本部分使用$\dfrac{4}{5}$的数据集作为训练集……
\end{lstlisting}

\begin{lstlisting}
% Mathtype转化出的行间数学公式Tex代码
即可得到:
\[a+b\leq c+d\]

% 正确修改为
即可得到:
\begin{equation}
    a+b\leq c+d
\end{equation}
\end{lstlisting}





\subsection{模型求解}



% ------------------------------------------------------------ %
%    打开chapter文件夹里的问题二建模.tex文件,编辑其中的内容
% ------------------------------------------------------------ %
%    打开chapter文件夹里的问题二建模.tex文件,编辑其中的内容
% ------------------------------------------------------------ %
\section{问题二建模与求解}

%\subsection{数据预处理}

\subsection{模型建立}

\subsection{模型求解}



% ------------------------------------------------------------ %
%    打开chapter文件夹里的问题三建模.tex文件,编辑其中的内容
% ------------------------------------------------------------ %
%    打开chapter文件夹里的问题三建模.tex文件,编辑其中的内容
% ------------------------------------------------------------ %
\section{问题三的建模与求解}

%\subsection{数据预处理}

\subsection{模型建立}

\subsection{模型求解}





% ------------------------------------------------------------ %
%    打开chapter文件夹里的模型评价.tex文件,编辑其中的内容
% ------------------------------------------------------------ %
%    打开chapter文件夹里的模型评价.tex文件,编辑其中的内容
% ------------------------------------------------------------ %
\input{chapter/9-模型评价.tex}



% ------------------------------------------------------------ %
%    打开chapter文件夹里的参考文献.tex文件,编辑其中的内容
% ------------------------------------------------------------ %
%    打开chapter文件夹里的参考文献.tex文件,编辑其中的内容
% ------------------------------------------------------------ %
\input{chapter/10-参考文献.tex}
\newpage



% ------------------------------------------------------------ %
%    打开chapter文件夹里的附录.tex文件,编辑其中的内容
% ------------------------------------------------------------ %
%    打开chapter文件夹里的附录.tex文件,编辑其中的内容
% ------------------------------------------------------------ %
% -----------------------------------------------------
% 下面的附录格式仅供参考,说不定大家到时候时间太紧都
% 没时间写附录了,附录这部分是最开放的哈,想怎么写怎
% 么写,主要内容就是代码和你想放在文章里又没放的东西,
% 比如大型图表等。
% -----------------------------------------------------

\appendix  % 别动!!!

\section{附录:基于Python的统计分析}

\subsection{描述性统计分析}


\begin{lstlisting}[language=Python]
print('Hello World!')
\end{lstlisting}



\subsection{数据可视化}


\begin{lstlisting}[language=Python]
import pandas as pd
import numpy as np
import matplotlib.pyplot as plt
import seaborn as sns

plt.rcParams["font.sans-serif"] = ["SimHei"]
plt.rcParams['font.size'] = 12  # 字体大小
plt.rcParams['axes.unicode_minus'] = False  # 正常显示负号

XXXXXX

plt.figure(figsize=(8, 7))
plt.scatter(df['item_price'], df['ord_qty'], s=10, c='red')
plt.xlabel('item_price')
plt.ylabel('ord_qty')
plt.title("产品价格-需求量散点图")
plt.show()
\end{lstlisting}


\section{附录:基于Python的统计推断}

\subsection{支持向量机}

\begin{lstlisting}[language=Python]
# 支持向量机的实现
\end{lstlisting}


\section{附录:图表(随便放的别当真)}

\begin{figure}[H] 
	\centering %图片居中
	\includegraphics[width=0.95\textwidth]{1.png} %插入图片,[]中设置图片大小,{}中是图片文件名
	\caption{附录图表1} %最终文档中希望显示的图片标题
	\label{Fig.main2} %用于文内引用的标签
\end{figure}

\begin{table}[H]
	\centering
	\begin{tabular}{c c} 
		\toprule[1.5pt]
		符号 & 含义  \\ 
		\midrule[1pt]
		$i=1,i=2$ & 分别表示高钾、铅钡玻璃 \\ 
		$j$ & 表示表中从二氧化硅($SiO_2$)到二氧化硫($SO_2$)中第$j$类化学物质 \\
		$z=1,z=2$ & 分别表示风化前和风化后 \\
		$x_1,x_2,x_3,x_4$ & 分别表示纹饰、类型、颜色、风化表面 \\
		$y_j$ & 表示第$j$类化学物质的含量 \\ 
		$\overline{y_j}$ & 表示第$j$类化学物质的平均含量 \\  
		\toprule[1.5pt]
	\end{tabular}
    \caption{附录图表1}
\end{table} 



\end{document} 
