\section{模型评价和改进}

\subsection{模型优点}

1. 在问题一中对文物表面是否风化与类型、颜色、纹饰关系分析过程中,不仅对于单变量之间进行了分析,还进一步用树模型进行了多变量与单变量的分析,同时利用互信息进一步对结果进行检验,提高模型的合理性。

2. 本文做了大量图表来统计分析数据特点,直观的对比出两类玻璃风化前后的化学成分的变化量以及各类玻璃化学成分之间的关系。

3. 在文本中多次对模型进行调参,利用混淆矩阵和多个指标检验,提高了模型的准确性。

4. 使用强分类器,构建Voting集成算法,得到一个完美模型,得到结果可信度高。

\subsection{模型缺点}

1. 做编码时由于颜色样本有7个非叙述类别,而数据集中存在大量的分类特征,没有找到合理高效的特征编码方式。

2. 本文在补充颜色缺失值时,鉴于分布分析补充黑色,而实际上在工业上常使用众数补充。

\subsection{模型改进}

在查阅文献时,涉猎了一种工业上常见新进的特征编码方式——目标编码,此方式通过计算类别出现的频率,对高维分类特征进行编码,避免出现维数灾难,同时也不会导致类别间的欧几里得距离过大。相信使用这种编码会让问题一中结果更具说服力。