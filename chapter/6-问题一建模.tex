\section{问题一的模型建立与求解}

%\subsection{数据预处理}

\subsection{模型建立}

模型建立部分是需要有大量的数学公式的,而很多小伙伴都没有 \LaTeX 基础,这里就不得不提到Mathtype了,队友可以先用Mathtype在Word或者WPS中敲好公式,然后再用Mathtype直接将公式转换为Tex代码。Mathtype安装、配置方法可以在我的个人网页中找到哈,仅供学习使用,不可用作商业用途!

值得注意的是这样转换而来的代码还是需要修改的,但是这个修改很容易!!!首先要注意,转换出来的代码全都是用:

\begin{lstlisting}
\[数学公式Tex代码\]
\end{lstlisting}

当然不排除我的Mathtype版本不对,正版、新版的已经修正。如果这样的话你需要把行内公式改为用一前一后的一对“\$”包裹,而行间公式则是用专门的数学环境——equation环境包裹。行内公式和行间公式示例如下:

\begin{lstlisting}
% Mathtype转化出的行内数学公式Tex代码
本部分使用\[\dfrac{4}{5}\]的数据集作为训练集……

% 正确修改为
本部分使用$\dfrac{4}{5}$的数据集作为训练集……
\end{lstlisting}

\begin{lstlisting}
% Mathtype转化出的行间数学公式Tex代码
即可得到:
\[a+b\leq c+d\]

% 正确修改为
即可得到:
\begin{equation}
    a+b\leq c+d
\end{equation}
\end{lstlisting}





\subsection{模型求解}