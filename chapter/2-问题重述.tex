\section{问题重述}

\subsection{问题背景}

数学建模比赛论文是要我们解决一道给定的问题,所以正文部分一般应从问题重述开始,一般确定选题后就可以开始写这一部分了。
这部分的内容是将原问题进行整理,将问题背景和题目分开陈述即可,所以基本没啥难度。
本部分的目的是要吸引读者读下去,所以文字不可冗长,内容选择不要过于分散、琐碎,措辞要精练。
注意:在写这部分的内容时,绝对不可照抄原题!(论文会查重)
应为:在仔细理解了问题的基础上\cite{ref1},用自己的语言重新将问题描述一遍。语言需要简明扼要,没有必要像原题一样面面俱到。


\subsection{问题提出}

看到上面插入的“\cite{ref1}”了吗,这是在引用后面的文献,所以大家在文中引用文献的时候也像我这样在句子结束后面加一个这个就行了,他是蓝色的,比较醒目,一般也都是蓝色的好像。

现有一组由专家给出的关于玻璃的数据,要求通过分析与建模解决下面若干问题:

\textbf{问题1:} 问题提出内容问题提出内容问题提出内容问题提出内容问题提出内容问题提出内容问题提出内容问题提出内容。

\textbf{问题2:} 问题提出内容问题提出内容问题提出内容问题提出内容问题提出内容问题提出内容问题提出内容问题提出内容问题提出内容。

\textbf{问题3:} 问题提出内容问题提出内容问题提出内容问题提出内容问题提出内容问题提出内容问题提出内容问题提出内容问题提出内容。

\textbf{问题4:} 问题提出内容问题提出内容问题提出内容问题提出内容问题提出内容问题提出内容问题提出内容问题提出内容问题提出内容。